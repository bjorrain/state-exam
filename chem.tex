\documentclass[10pt,a4paper,twocolumn,landscape]{article}

\usepackage[english,russian]{babel}
\usepackage{amsmath, unicode-math}
\usepackage{soul}
\usepackage{pgf,tikz,pgfplots}
\usepackage{wrapfig,graphicx}
\usepackage{fancyhdr,lastpage,array}
\usepackage{multirow,multicol,enumitem}
\usepackage[top=2.5cm, bottom=2cm, left=2.2cm, right=1.5cm]{geometry}
\usepackage{pgffor}
\usepackage{makecell}
\usepackage{qrcode}
\usepackage{chemfig}
\usepackage{firamath-otf}
\usepackage{mathastext}

% Set fonts
%\setmainfont{Times New Roman}
\setmainfont[Scale=1]{PT Serif}
\setsansfont{PT Sans}
%\usepackage[euler-digits,euler-hat-accent]{eulervm}
\usepackage[mathscr]{euscript}

\setlist[enumerate]{nosep}

\everymath{\displaystyle}
\newcommand{\Z}{\mathcal{Z}}
\newcommand{\Q}{\mathcal{Q}}
\newcommand{\N}{\mathcal{N}}
\newcommand{\No}{\mathcal{N}_0}
\newcommand{\R}{\mathcal{R}}

\pgfplotsset{compat=newest}

\setlength{\parindent}{0cm}

\setlength{\columnsep}{2.5cm}
\setlength{\columnwidth}{11.3cm}

\renewcommand{\headrulewidth}{0cm}
\renewcommand{\footrulewidth}{0cm}
\renewcommand{\headruleskip}{0.6cm}
\renewcommand{\footruleskip}{0cm}

\newcommand{\PreserveBackslash}[1]{\let\temp=\\#1\let\\=\temp}
\newcolumntype{C}[1]{>{\PreserveBackslash\centering}p{#1}}
\newcolumntype{R}[1]{>{\PreserveBackslash\raggedleft}p{#1}}
\newcolumntype{L}[1]{>{\PreserveBackslash\raggedright}p{#1}}


% \createpart{2} -> Часть 2
\newcommand{\createpart}[1]{ \center{\textsf{\textbf{Часть #1}}} }

% \createtask{2}{text} -> [2] text
\newcommand{\createtask}[2]{
	\hspace{-1.3cm}
	\begin{tabular}{c l}
		\framebox[0.85cm]{\textsf{\textbf{#1}}}\hspace{-0.15cm} &
		\begin{minipage}[t]{11.25cm}
			#2
		\end{minipage}
	\end{tabular}
	\phantom{1}
}

% \blankanswer -> [4.5cm] line
\newcommand{\blankanswer}{ \underline{\hspace{4.5cm}} }

% \partbox{text} -> [text]
\newcommand{\partbox}[1]{
	\hspace{-0.3cm}
	\framebox[11.5cm]{
		\begin{minipage}[l]{11.2cm}
			\vspace{-1em}\phantom{1}\\
			\textsf{\textit{\textbf{ #1 }}}
			\vspace{-0.1cm}
		\end{minipage}
	}
	\phantom{1}
}

\newcommand{\taskbox}[1]{
	\hspace{-0.3cm}
	\framebox[11.5cm]{
		\begin{minipage}[l]{11.2cm}
			\vspace{-1em}\phantom{1}\\
			\textsf{ #1 }
			\vspace{-0.1cm}
		\end{minipage}
	}
	\phantom{1}
}	
% \createheaders -> set up fancyhdr
\makeatletter
\newcommand{\createheaders}{
	\raggedbottom
	\pagestyle{fancy}
	\fancypagestyle{plain}{}
	\setlength{\headheight}{0cm}
	\fancyhead[L]{
		\scriptsize
		\textsf{ 
			\begin{tabular}{L{4.2cm} R{4.2cm} R{1.9cm}}
				\hspace{-0.1cm}\@title & \@author & \textsf{\thepage}/\textsf{\pageref{LastPage}}
			\end{tabular}
		}
	}
	\fancyhead[R]{\textsf{\scriptsize \@title\hspace{0.4cm}}}
	\fancyfoot[L]{
		\hspace{0.5cm}
		\begin{minipage}[l]{11.5cm}
			\center{
				\textsf{ \tiny
					© 2024 Федеральная служба по надзору в сфере образования и науки
					\\[0.02cm]
					Копирование \textbf{не допускается}
				}
			}
		\end{minipage}
	}
	\fancyfoot[C]{}
}
\makeatother

% \separ -> 1cm
\newcommand{\separ}{\vspace{1em}}

% \sepline -> 1em
\newcommand{\sepline}{\vspace{1em}}



\title{\qrcode[height=1cm]{https://fourgia.t.me/} \textsf{ОТКРЫТЫЙ ВАРИАНТ}}
\author{\textsf{ХИМИЯ -- 11 КЛАСС}}


\begin{document}

\begin{center}
	\textsf{\textbf{Единый государственный экзамен по ХИМИИ \\
	Инструкция по выполнению работы \\}}
	\sepline
	Экзаменационная работа состоит из двух частей, включающих в себя
34 задания. Часть 1 содержит 28 заданий с кратким ответом, часть 2
содержит 6 заданий с развёрнутым ответом. \\
На выполнение экзаменационной работы по химии отводится 3,5 часа
(210 минут). \\
Ответом к заданиям части 1 является последовательность цифр или
число. Ответ запишите по приведённым ниже образцам в поле ответа
в тексте работы, а затем перенесите в бланк ответов № 1.
Последовательность цифр в заданиях 1–28 запишите без пробелов, запятых
и других дополнительных символов. \\

\taskbox{
\text{Ответ в КИМ:}
\begin{tabular}{|c|c|}
	\hline
	3 & 5 \\
	\hline
\end{tabular} \text{Ответ в бланке:} \begin{tabular}{|c|c|c|c|c|}
	\hline
	3 & 5 & & &\\
	\hline
\end{tabular} \\ 
\text{Ответ в КИМ:}
	\begin{tabular}{|c|c|}
		\hline
		X & Y \\
		\hline
		3 & 5 \\
		\hline
	\end{tabular} \text{Ответ в бланке:} \begin{tabular}{|c|c|c|c|c|}
		\hline
		3 & 5 & & &\\
		\hline
		\end{tabular} \\
\text{Ответ в КИМ:} \underline{   3,5    }
\text{Ответ в бланке:} \begin{tabular}{|c|c|c|c|c|}
			\hline
			3 & , & 5 & &\\
			\hline
			\end{tabular} \\
}
\sepline \\
Ответы к заданиям 29–34 включают в себя подробное описание всего хода выполнения задания. В бланке ответов № 2 укажите номер задания
и запишите его полное решение. \\
Все бланки ЕГЭ заполняются яркими чёрными чернилами.
Допускается использование гелевой или капиллярной ручки. \\
При выполнении заданий можно пользоваться черновиком. \textbf{Записи
в черновике, а также в тексте контрольных измерительных материалов
не учитываются при оценивании работы.} \\ 
При выполнении работы используйте Периодическую систему химических элементов Д.И. Менделеева, таблицу растворимости солей, кислот и оснований в воде, электрохимический ряд напряжений металлов. Эти сопроводительные материалы прилагаются к тексту работы.\\ 
Для вычислений используйте непрограммируемый калькулятор. \\
Баллы, полученные Вами за выполненные задания, суммируются. Постарайтесь выполнить как можно больше заданий и набрать наибольшее количество баллов. \\
После завершения работы проверьте, чтобы ответ на каждое задание в бланках ответов № 1 и № 2 был записан под правильным номером. \\
\textit{\textsf{\textbf{Желаем успеха!}}}
\end{center}
\createheaders

\createpart{1} \\ \sepline
\partbox{
	Ответами к заданиям 1--28 являются число или последовательность
	цифр. Ответ запишите в поле ответа в тексте работы, а затем
	перенесите в БЛАНК ОТВЕТОВ № 1 справа от номера
	соответствующего задания, начиная с первой клеточки. Каждый символ
	пишите в отдельной клеточке в соответствии с приведёнными в бланке
	образцами. Цифры в заданиях 7, 8, 10, 14, 15, 19, 20, 22, 23, 24, 25 могут повторяться.
}
\separ

\taskbox{\partbox{Для выполнения заданий 1-3 используйте следующий ряд элементов: \\
			1) Mn, 2) Cu, 3) In, 4) Cr, 5) Se 			
}}
\separ
\\
\foreach \num in {1, 2, ..., 28}{
\createtask{\num}{\input{chem/task-\num}}\\}

\separ

\separ

\partbox{
	Не забудьте перенести все ответы в бланк ответов № 1
в соответствии с инструкцией по выполнению работы.
Проверьте, чтобы каждый ответ был записан в строке с номером
соответствующего задания.
}

\pagebreak

\createpart{2}
\\ \sepline 
\partbox{
	Для записи ответов на задания 29--34 используйте БЛАНК
	ОТВЕТОВ № 2. Запишите сначала номер задания (29, 30 и т.д.), а затем
	решение соответствующей задачи. Ответы записывайте чётко
	и разборчиво.	
}
\separ
\\
\taskbox{Для выполнения заданий 29 и 30 используйте следующий перечень веществ. 
\begin{enumerate}
    \item азотная кислота
    \item свинец
    \item соляная кислота
    \item серная кислота
    \item сероводород
    \item нитрат калия
    \item гексакобальтинитрит натрия
\end{enumerate}
Допустимо использование водных растворов.} \\
\separ
\foreach \num in {29, 30, ..., 34}{
\createtask{\num}{\input{chem/task-\num}}\\}

\partbox{Проверьте, чтобы каждый ответ был записан рядом с номером
соответствующего задания.}
\end{document}

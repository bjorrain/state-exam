Установите соответствие между уравнением химической реакции и направлением смещения химического равновесия при увеличении давления в системе: к каждой позиции, обозначенной буквой, подберите соответствующую позицию, обозначенную цифрой. \\

\begin{tabular}{|c|c|}
    \hline
    \textsf{\textbf{Уравнение реакции}} & \makecell{\textbf{\textsf{Направление смещения}} \\ \textbf{\textsf{химического равновесия}}} \\
    \hline
    А. \makecell{\text{$N_2O_3 \leftrightharpoons NO + NO_2$}} & 1. смещается в сторону продуктов реакции\\
    Б. \makecell{\text{$N_2O_4 \leftrightharpoons 2 NO_2$}} & 2. смещается в сторону исходных веществ\\
    В. \makecell{\text{$N_2O_5 + H_2O \leftrightharpoons 2 HNO_3$}} & 3. не происходит смещения равновесия\\
    Г. \makecell{\text{$H_2 + I_2 \leftrightharpoons 2 HI$}} & \\
    Д. \makecell{\text{$H_2 + Cl_2 \leftrightharpoons 2 HCl$}} & \\
    \hline
\end{tabular} \\

\sepline
Ответ: \\
\begin{tabular}{|c|c|c|c|c|}
\hline
А) & Б) & В) & Г) & Д) \\
\hline
__ & __ & __ & __ & __\\
\hline
\end{tabular}
Установите соответствие между уравнением окислительно-восстановительной реакции и суммой коэффициентов перед реагентами в ней. К каждой позиции, обозначенной буквой, подберите соответствующую позицию, обозначенную цифрой.

\begin{tabular}{|c|c|}
    \hline
    \textsf{\textbf{уравнение}} & \textsf{\textbf{сумма коэффициентов}} \\
    \hline
    \makecell{А. \text{$I_2 + F_2 \to IF_5$}} & \makecell{1. \text{11}}\\
    \makecell{Б. \text{$P + O_3 \to P_2O_5$}} & \makecell{2. \text{6}}\\
    \makecell{В. \text{$KMnO_4 + H_2O_2 \to MnO_2 + O_2 + KOH + H_2O$}} & \makecell{3. \text{5}}\\
    \makecell{Г. \text{$H_2S + Cl_2 \to S + HCl$}} & \makecell{4. \text{2}}\\
    \makecell{Д. \setchemfig{atom sep=1em,bond style={line width=1pt 2pt}} \chemfig{*6(--=---)} $+ H_2O_2 \to $ \chemfig{*6(---(-OH)-(-OH)--)}}  & \makecell{5. \text{7}} \\
    \hline
    \end{tabular}
\\
\sepline
Ответ: \\
\begin{tabular}{|c|c|c|c|c|}
\hline
А) & Б) & В) & Г) & Д) \\
\hline
__ & __ & __ & __ & __\\
\hline
\end{tabular}
При сжигании 100г органического вещества А образуется 122.74л (н.у.) углекислого газа и 110.96г воды. Также известно, что в присутствии кислоты вещество А перехдит в вещество Б, молярная масса которого на 12\% отличается от массы вещества А. Определите молекулярную формулу вещества А, установите его структуру и напишите уравнение его превращения в вещество Б.
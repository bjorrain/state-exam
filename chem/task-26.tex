Медный купорос массой 25 граммов растворили в 250 мл воды температурой 20 градусов Цельсия. Потом этот раствор нагрели на газовой горелке до кипения и выпаривали 5 минут. Какова массовая доля сульфата меди в полученном растворе?
\\
(считайте температуру кипения раствора равной 100.19 градусам Цельсия, теплоемкость раствора - 4.2 кДж/$^\circ C$*кг, теплоту сгорания метана - 896 кДж/моль. за 1 минуту тратится 3 литра метана, потери тепла считать равными 70\%)

\\
Ответ: \blankanswer. \\
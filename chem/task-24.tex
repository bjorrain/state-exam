Установите соответствие между формулами веществ и реагентом, с помощью которого их можно различить: к каждой позиции, обозначенной буквой, подберите соответствующую позицию, обозначенную цифрой.\\

\begin{tabular}{|c|c|}
    \hline
    \textsf{\textbf{формулы веществ}} & \textsf{\textbf{реагенты}} \\
    \hline
    \makecell{А.\text{оксид хрома (VI) и сульфид меди (II)}} & \makecell{1. \text{иодная вода}}\\
    \makecell{Б.\text{иодид калия и фосфат натрия}} & \makecell{2. \text{нитрат серебра}}\\
    \makecell{В.\text{хлорид магния и хлорид ванадия (III)}} & \makecell{3. \text{перманганат натрия}}\\
    \makecell{Г.\text{диметиламин и анилин}} & \makecell{4. \text{азотистая кислота}}\\
    \makecell{Д.\text{ацетон и изопропиловый спирт}} & \makecell{5. \text{алюминон}} \\
    \hline
\end{tabular}
\\
\sepline
Ответ: \\
\begin{tabular}{|c|c|c|c|c|}
\hline
А) & Б) & В) & Г) & Д) \\
\hline
__ & __ & __ & __ & __\\
\hline
\end{tabular}
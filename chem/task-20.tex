Установите соответствие между формулой вещества и продуктом, образующимся на катоде при электролизе его водного раствора: к каждой позиции, обозначенной буквой, подберите соответствующую позицию, обозначенную цифрой.

\begin{tabular}{|c|c|}
    \hline
    \textsf{\textbf{формула вещества}} & \textsf{\textbf{продукт на катоде}} \\
    \hline
    \makecell{А.\text{$FeCl_3$}} & \makecell{1. \text{$Fe, Fe(OH)_3, Fe_2O_3, H_2$}}\\
    \makecell{Б.\text{$MnCl_2$}} & \makecell{2. \text{$Fe, H_2$}}\\
    \makecell{В.\text{$CuSO_4$}} & \makecell{3. \text{$H_2$}}\\
    \makecell{Г.\text{\setchemfig{atom sep=1em,bond style={line width=1pt 2pt}} \chemfig{-[0.5]-([2]=O)-[-0.5]ONa}}} & \makecell{4. \text{ \setchemfig{atom sep=1em,bond style={line width=1pt 2pt}}\chemfig{-[0.5]-[-0.5]-[0.5]-}}}\\
    \makecell{Д.\text{HCl}} & \makecell{5. \text{$Mn, MnO, MnO_2, Mn(OH)_2, H_2$}} \\
    \makecell{} & \makecell{6. \text{$Mn, H_2$}} \\
    \hline
\end{tabular}


Ответ: \sepline \\
\begin{tabular}{|c|c|c|c|c|}
\hline
А) & Б) & В) & Г) & Д) \\
\hline
\makecell{} & \makecell{} & \makecell{} & \makecell{} & \makecell{}\\
\hline
\end{tabular}
